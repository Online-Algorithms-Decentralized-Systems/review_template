\documentclass{article}
\usepackage{graphicx} % Required for inserting images
\usepackage{authblk}
\usepackage[utf8]{inputenc}
\usepackage[english]{babel}
\usepackage[T1]{fontenc}
\usepackage{geometry}
\geometry{margin=1in}

\usepackage{amsmath}
\usepackage{algorithm}
\usepackage{algpseudocode}
\usepackage{amsthm}
\usepackage{amssymb}
\usepackage{graphicx}
\usepackage{nameref}
\usepackage{placeins}
\usepackage{hyperref}
\usepackage{biblatex}
\usepackage[dvipsnames]{xcolor}
\hypersetup{
    colorlinks,
    linkcolor={blue!80!black},
    citecolor={blue!80!black},
    urlcolor={blue!80!black}
}

\usepackage{lipsum}
\usepackage{enumitem}

% Command for writing comments
\newcommand\JC[1]{{\color{Maroon} JC: #1}}         % Jared:    Use \JC{my note}

\newcommand*{\doi}[1]{DOI: \href{http://dx.doi.org/#1}{#1}}\setitemize{noitemsep,topsep=5pt,parsep=2pt,partopsep=0pt}

% bibliography
\addbibresource{refs.bib}


% SETUP
\newtheorem{theorem}{Theorem}[section]
\newtheorem{corollary}{Corollary}[theorem]
\newtheorem{lemma}[theorem]{Lemma}
\newtheorem*{problem}{Problem}

\title{Review: Title of Paper}
\author[1]{Author Names}
\affil[1]{Loyola Marymount University}

\date{\today}


\begin{document}
\maketitle


\section{First Pass}
In this section, fill out each of the following questions (from~\cite{how_to_read_a_paper}) to provide a high-level overview of the paper.
This section does not need to be in prose format.
\begin{enumerate}
    \item Category: What type of paper is this? A measurement paper? An analysis of an existing system? A description of a research prototype?
    \item Context: Which other papers is it related to? Which theoretical bases were used to analyze the problem? Which experimental methods were used to evaluate the results?
    \item Correctness: Do the assumptions appear to be valid? Are the theorems correct? Is the experimental methodology correct?
    \item Contributions: What are the paper's main contributions?
    \item Clarity: Is the paper well written?
\end{enumerate}

\section{Second Pass}
In this section, provide a concise summary of the paper, capturing its main contributions, methodology, and findings. The summary (written in prose) should address the following:

\begin{itemize}
    \item \textbf{Objective:} What problem or question is the paper addressing, and why is it significant?
    \item \textbf{Approach:} What methods or frameworks does the paper employ to address its objective?
    \item \textbf{Results:} What are the key findings or outcomes of the research?
    \item \textbf{Implications:} How do the results contribute to the field, and what are their potential applications or future directions?
\end{itemize}

This section should be comprehensive enough to provide a clear understanding of the paper's core content without requiring the reader to refer back to the original text.

\section{Third Pass}
In this section, provide a detailed evaluation of the paper, focusing on its strengths and weaknesses. Consider the following aspects:

\begin{itemize}
    \item \textbf{Strengths:} What aspects of the paper are particularly well-presented? This may include the novelty of its contributions, the rigor of its methodology, or the clarity of its presentation.
    \item \textbf{Weaknesses:} What are the limitations or shortcomings of the paper? Consider gaps in the research, weaknesses in the argumentation, or issues with reproducibility.
\end{itemize}

Additionally, this section should discuss \textit{new} related works that have been published after the subject paper’s release. Explore how these newer works build upon, challenge, or expand the original paper's findings. 

Use proper citations formatted with BibTeX (referencing `refs.bib') to incorporate these additional sources, like this~\cite{how_to_read_a_paper}.


\printbibliography

\end{document}
